\documentclass[11pt]{article}
\usepackage[utf8]{inputenc}
\usepackage[left=4cm, right=4cm, top=2cm]{geometry}
\usepackage[none]{hyphenat}
\title{\textbf{Dynamic Capabilities: What are they and how can they support your strategy?} \newline Successful vs. unsuccessful implementations}


\date{}
\sloppy


\begin{document}
\author{David Daogaru}
\maketitle

\section{Introduction}
How can you develop and exploit dynamic capabilities (DCs) in your firm? What are the factors that indicate and influence your strategy? When and how can you drive strategy by making use of dynamic capabilities? These are the questions that we will focus on in an attempt to explain the relevance of the business model, the organizational design and agility, the external and internal environment and how they can provide valuable information to leaders regarding the dynamic capabilities that they need to address in order to succeed on the market.

\section{The main problem: Handling dynamic capabilities in the organization in an appropriate manner}
Although some academics define DCs as 'organizational and strategic routines by which firms achieve new resource configurations' (Eisenhardt and Martin, 2000), others prefer to express them as capacities to sense, shape and seize opportunities (Felin and Powell, 2016) or even add the 'transforming needed to design and implement a business model' to these capacities (Teece, 2018). We will consider dynamic capabilities in the broadest sense here, including the classification made by Ambrosini, Bowman \& Collier on three levels: incremental (routinized development of the resource base), renewing (changing and expanding the resource base) and regenerative (developing new sets of DCs in a "cyclic" manner). There are also other forms of describing them, such as ordinally, distinguishing between ordinary capabilities, or "microfoundations", involving the 'adjustment and recombination of [...] existing capabilities as well as the development of new ones and higher-order DCs, such as creating new products or expanding in new areas (e.g. via acquisitions and alliances) (Teece, 2018). We will consider most ways of presenting them in order to focus on how to successfully apply them in the strategy of the firm and what to avoid. We will therefore think of DCs as means of creating and capturing wealth (Teece, Pisano \& Shuen, 1997) by shaping and reconfiguring the competences of a firm, thus developing competitive advantage (Leonard-Barton, 1992). Now that we have an idea of what DCs are, we will discuss their relevance, filtering them through the "umbrella" of factors that they influence in an organization. \newline \newline
\textbf{People, Culture \& Structure} \newline \newline
Dynamic capabilities in the form of individual and group aspects (the expertise of the members, the synergy of the teams, the competencies of the leaders etc.) have a strong impact on the performance of the organization, such that they define and shape the environment, determine the direction of firm and even generate its power. The main concern we have relating to the social aspect of the firm is the \emph{balance} between rigidity and flexibility. This equilibrium is essential for progress, as flexibility is the only aspect to make room for innovation, while rigidity (hierarchy and authority) is vital for control and order maintaining. While command and control is easily implemented, maintained and even quantified for measuring progress, the success trap of familiarity (Wang, Senaratne \& Rafiq, 2014) can easily lead to monotony and even failure, becoming incapable of responding to the volatile and disruptive market. This was precisely the case of Nokia, which dramatically lost its leading position in the market to rivals such as Apple and Samsung. Nokia was competitive and focused on the external environment (the major shift to advanced operating systems developing ecosystems such as the App Store and the Android Market - now Google Play) by having its own OS - Symbian - with similar promising capabilities. However, the rigidity of the organizational climate -  in which the main target was high performance - put pressure and strong emphasis on the management, so that they ended up convincing/misleading seniors into thinking that the progress is just as desired by filtering information, hiding concerns and being unrealistic only to follow the rules and to obtain the promised rewards (Thompson et al., 2017). 
\newline \newline
\textbf{Assets, Processes \& Systems} \newline \newline
Regarding the "non-human" DCs, we think of influence/expansion, adapting/renewing processes, decisiveness and responsiveness etc. labelled as higher-order DCs (Teece, 2018). These determine the ability of a firm to respond to inevitable market fluctuations and uncertainty. Specifically, the need of appropriate power through assets or influence is exemplified by the harsh loss the EMI Group (conglomerate) experienced with the CT scanner innovation that one engineer made through extensive research. Unfairly considered as a strategical failure, the misfortune was caused by rivals who replicated their scanner and by the time EMI developed the capabilities to exploit it, the market was already dominated by GE and Siemens (Kleiner, 2018) and EMI later dropped out of the CT scanner business (Teece, 1986) ending up with nothing but copyright infringement lawsuits. The severe lack of DCs required for a successful transition to the US market - manufacturing and marketing adaptability - is what deprived them of success and their incapacity to build difficult-to-replicate distinctive competencies (Teece, Pisano \& Shuen, 1997) made them a vulnerable target for rivals.

\newline \newline

\section{Solutions: Actively engaging in strategy development via dynamic capabilities}
\subsection{Solutions for the "human" element}
We will focus first on the aspect of people and their social dynamics and propose solutions for enhancing the social interaction, cohesion and integration via organizational design. The following suggestions are critical for a superior organizational (internal) coherence:
\newline \newline
\emph{Allow and empower flexibility.} Companies must learn to become flexible in order to break the chains of organizational inertia (Teece, 2018) by enabling strategic renewal (Agarwal \& Helfat, 2009). The dynamic process of integration is what enables firms to create a new resource base (Ambrosini, Bowman \& Collier, 2009) based on agility, innovation and adaptation (Felin \& Powell, 2016). The standard rewarding system for following the rules and engaging in normalized activity should be boosted with incentives for innovation, creativity and self-directed work. This is the case of Valve Corporation, which gives a very high degree of freedom to its employees, leveraging their flexibility and creativity (Valve Handbook for New Employees, 2018). The implementation of such a polyarchic structural model enabled Valve to become a leader in volatile markets, to develop, publish and distribute best sellers and to evolve into a platform, transition which Nokia failed deplorably. In Valve's case, the one who responds is the one who is the closest to action, minimizing the impediments of bureaucracy and hierarchy (Felin \& Powell, 2016). The benefits are multiple and the DCs are regenerative in the sense that they determine employees to constantly grasp and share content, thus learning how to learn. The limitations of this solution reside in the people and the culture, so that organizational chaos is a risk in the case of too much freedom with too little governance, particularly the problem of imbalance between differentiation and integration (Felin \& Powell, 2016). This leads to the second suggestion for the "human" aspect: \newline \newline
\emph{Engender homogeneity and integration.} Organizations tend to perform better in an integrated environment and this requires DCs in order to enhance the cohesion of the people, particularly when barriers are in your way. These include: communication and coordination inconsistencies, diversity and distance (as seen in international expansion), organizational differentiation (different or irreconcilable practices). This was the case of the development of Windows Vista, project which required thousands of developers across the globe to complete it (Bird, Nagappan, Devanbu, Gall \& Murphy, 2009). The study of Bird et al. found that the binaries engineers developed in different buildings 'have, on average, 16\% more post-release failures than binaries developed in the same building'. One finding was that the cultural barriers diminish when extended visits are made, due to the reinforcement of trust and promptness. Furthermore, they recommend synchronous communication via common schedules, consistent use of resources and minimization of ownership changes for various pieces of work for organizational integration, claiming that the manager can maintain the organizational culture by coordination, thus reducing the otherwise geographical differences. Their conclusion, reinforced by their previous study (Nagappan, Murphy \& Basili, 2008), is that 'organizational differences' are much stronger indicators of quality than geographical ones'. This form of enhancement is limited by groups with strong subcultures that may form collective resistance against integration (Jeanes, 2017). The possible drawback of successfully implementing integration consists of group bias (i.e. group think or group polarisation), when members are prone to negatively change their perception and behaviour towards work-related matters (Jeanes, 2017). To avoid this, the manager must act sensibly on the group formation and integration so that no discrepancies are ignored. \newline \newline
For the solutions provided to the "human" element, the manager must also be critical in approaching them. For a company with predominantly automatized activities/processes it makes more sense to maintain a more bureaucratic and less flexible style, ensuring their constant enhancement via first-order DCs which incrementally improve the resource base. Flexibility can still be beneficial, but the extent depends on both the external and the internal environment and the manager can handle the "ratio" between the two. Similarly, integration is more likely to be applicable to companies which make good use of renewing DCs in a rapidly changing or expanding environment and, again, the executives/managers have the required knowledge of these aspects ahead of time.

\subsection{Solutions for the "non-human" element}
In terms of the technical aspects approached above (assets, processes \& systems), the DCs desired are responsiveness and resilience among teams as well as expansion and diversification on the market for a stronger competitive ability. We suggest the following: \newline \newline
\emph{Cultivate perseverance even in times of success}. The reason for this is to achieve a degree of stability and certainty in progress. King Digital Entertainment is a highly successful leader on the video game market by maintaining its position through frequent breakthroughs. In spite of creating incredibly popular games, such as Candy Crush and Farm Heroes, regularly boosting revenues, the CEO's strategy is to 'build a portfolio of games' which address the market requests while still being more attractive choices than traditional games due to the enhanced quality. Moreover, King is leveraging the data about customer habits, securing a continuous development on its games by analyzing the consumers' behaviour and responding appropriately and regularly through updates (Thompson et al., 2017). The response to customer needs is determined by the strength of the DCs King possesses.
\newline\newline
\emph{Always look for expansion and diversification}. Decisive DC in most markets, augmentation complements the integration discussed earlier. A great example is constituted by Dorset Cereals, a breakfast food manufacturer which never ceased to evolve during its 29 years of existence. Throughout its lifetime, it endured multiple changes of leadership and subsequently strategy fluctuations, made contracts with retailers, won awards for exporting achievements, expanded and enlarged manufacturing capacities and covered many overseas markets. Regardless of the case, Dorset never failed to respond to the external and internal environment. It redesigned its website and produced new communication materials such as packaging to attract customers, it expanded services via hotels and catering services, it accelerated product development to respond to increasingly consuming markets and it implemented control systems to strengthen the internal power (Janes, 2017).
Changing and adapting to the environment were the main DCs Dorset demonstrated ever since it was founded and this was made via incremental development as well as renewal and regenerative abilities. The threats of this approach is that the company enters a loop of constant requests and responses (more power leads to more competition) and interrupting the cycle can shortly disrupt its entire activity. The leaders must always be vigilant while engaging in such a challenge.

\section{Conclusion}
Albeit the solutions proposed regard different aspects or "umbrellas" of factors in a company, they can be considered simultaneously for glorious success. However, there might arise tensions between the two due to a lack in resources, difficulties in the strategy or incompatibility with the industry. In order to pursue a positive outcome, the manager must critically evaluate the internal and external climate and must respond sensibly to the environmental demands. The (senior) management is sufficiently understanding and controlling the dynamic capabilities of an organization and the change should start with them.
\newpage
\section{References}
Agarwal, R., \& Helfat, C. (2009). Strategic Renewal of Organizations. Organization Science, 20(2), 281-293. doi: 10.1287/orsc.1090.0423
\newline \newline
Ambrosini, V., Bowman, C., \& Collier, N. (2009). Dynamic Capabilities: An Exploration of How Firms Renew their Resource Base. British Journal Of Management, 20, S9-S24. doi: 10.1111/j.1467-8551.2008.00610.x
\newline \newline
Bird, C., Nagappan, N., Devanbu, P., Gall, H., \& Murphy, B. (2009). Does distributed development affect software quality?. Communications Of The ACM, 52(8), 85. doi: 10.1145/1536616.1536639
\newline \newline
Eisenhardt, K., \& Martin, J. (2000). Dynamic capabilities: what are they?. Strategic Management Journal, 21(10-11), 1105-1121. doi: 10.1002/1097-0266(200010/11)21:10/11<1105::aid-smj133>3.0.co;2-e
\newline \newline
Felin, T., \& Powell, T. (2016). Designing Organizations for Dynamic Capabilities. California Management Review, 58(4), 78-96. doi: 10.1525/cmr.2016.58.4.78
\newline \newline
Janes, A. (2017). Case Study: Dorset Cereals [PDF] (pp. 10-33). University of Exeter.
\newline \newline
Jeanes, E. (2017). Groups. Organisational Behaviour.. Presentation, University of Exeter.
\newline \newline
Jeanes, E. (2017). Conflict and Resistance. Presentation, University of Exeter.
\newline \newline
Kleiner, A. (2018). The Dynamic Capabilities of David Teece. Retrieved from https://www.strategy-business.com/article/00225?gko=d24f3
\newline \newline
Leonard-Barton, D. (1992). Core capabilities and core rigidities: A paradox in managing new product development. Strategic Management Journal, 13(S1), 111-125. doi: 10.1002/smj.4250131009
\newline \newline
Nagappan, N., Murphy, B., \& Basili, V. (2008). The Influence of Organizational Structure on Software Quality: An Empirical Case Study. University Of Maryland, Department Of Computer Science, 1-9. Retrieved from https://www.cs.umd.edu/~basili/publications/proceedings/P125.pdf
\newline \newline
Teece, D. (1986). Profiting from technological innovation: Implications for integration, collaboration, licensing and public policy. Research Policy, 15(6), 285-305. doi: 10.1016/0048-7333(86)90027-2
\newline \newline
Teece, D. (2018). Business models and dynamic capabilities. Long Range Planning, 51(1), 40-49. doi: 10.1016/j.lrp.2017.06.007.
\newline \newline
Teece, D., Pisano, G., \& Shuen, A. (1997). Dynamic capabilities and strategic management. Strategic Management Journal, 18(7), 509-533. doi: 10.1002/(sici)1097-0266(199708)18:7<509::aid-smj882>3.0.co;2-z
\newline \newline
Thompson, A., Strickland III, A., Janes, A., Sutton, C., Peteraf, M., \& Gamble, J. (2017). Crafting and Executing Strategy: The Quest For Competitive Advantage (2nd ed., pp. 32-33). London: McGraw-Hill.
\newline \newline
Thompson, A., Strickland III, A., Janes, A., Sutton, C., Peteraf, M., \& Gamble, J. (2017). Crafting and Executing Strategy: The Quest For Competitive Advantage (2nd ed., pp. 148-150). London: McGraw-Hill.
\newline \newline
Valve Corporation. (2018). Valve Handbook for New Employees [Ebook]. Retrieved from \newline https://media.steampowered.com/apps/valve/Valve\_Handbook\_LowRes.pdf
\newline \newline
Wang, C., Senaratne, C., \& Rafiq, M. (2014). Success Traps, Dynamic Capabilities and Firm Performance. British Journal Of Management, 26(1), 26-44. doi: 10.1111/1467-8551.12066


\end{document}
